\section{Answers to questions}

\paragraph{}
\textbf{Q1:}
\textit{How does the .p12 file work? Explain what this file is and its relationship to the NTNU CA.}

The .p12-file is an archive file format containing a bundle of cryptography objects, defined by the PKCS\#12 RFC\cite{pkcs-12}.
It may contain x.509 public-key certificates\cite{x509}, encrypted private keys, certificate revocation lists, and more.
When imported into a web-browser like Mozilla Firefox, it can be used for establishing identity against servers requiring it.

Its relation to the NTNU CA is that the validity of the public key certificate contained within is guaranteed by a chain of CAs signing each others certificates, with the NTNU CA at the root.
Therefore, the public key certificate is only valid for entities having accepted the NTNU CA as their lord and saviour.

\paragraph{}
\textbf{Q2:}
\cprotect\textit{Explain what the string \verb/TLS_DHE_RSA_WITH_AES_128_CBC_SHA/ means in Figure 4. Comment on security related issues regarding the cryptographic algorithms used to generate and sign your group's web server certificate (key lengths, algorithms, etc.).}

The string \verb/TLS_DHE_RSA_WITH_AES_128_CBC_SHA/ describes what type of SSL cipher is used for the communication with the web server.
This security lives on the transport layer and is referred to as TLS.
The cipher used is AES, which was first introduced in 2001\cite{rfc3268} to modernize the TLS cipher suites.
Diffie Hellman is used for key exchange.
SHA is for message signing.

The choice of algorithm to use to generate and sign web certificates plays a huge part in security.
If one uses an unsecure algortihm like MD5 for signing one can create a rogue CA and thereby impersonate any website on the internet\cite{md5-harmful}.
As well as having a seucre algorithm, the key length must be long enough to make sure people can't break the security.
In the practical, AES with few bits was broken, even though AES itself is concidered secure\cite{nsa-secure}.

\paragraph{}
\textbf{Q3:}
\textit{List and explain the verifications. What is the difference between the two files? Who has signed the Apache release? What is the point of this process?}

Apache distributes all its releases with a corresponding OpenPGP signature\cite{pgp} and MD5 hash.
After retrieving the RSA key ID of the signature with GPG\cite{gpg}, the corresponding public key can be retrieved from a key server (pgpkeys.mit.edu for instance).
If the received public key matches the signature, then the file has not been tampered with since signing.
To verify that the received public key actually belongs to the person it claims to, a sufficiently secure in-person key exchange is required.
Alternatively, one can enter a web of trust.

The httpd-2.4.12.tar.gz.asc file contains the PGP Signature corresponding to the tar.gz archive.

The provided release (httpd-2.4.12.tar.gz) has been signed by Jim Jagielski with RSA key ID \verb/791485A8/.
This process ensures that we can verify the integrity and validity of httpd even when it has been acquired through a non-secure channel (http).

\paragraph{}
\textbf{Q4:}
\textit{What is the purpose of a certificate chain? Describe, on a high-level, the steps that your browser takes when verifying the authenticity of a web page served over HTTPS.}

Certificate chains are needed to verify that the public key, and other data, of a certificate is correct.
The browser verifies the authenticity of a host by looking at the signature of each certificate in the chain.
The signature has been created with the private key of the next certificate in the chain, and may therefore be verified using the public key of the next certificate.
The browser verifies each certificate in the list in order until it reaches a known and trusted certificate – a root CA.\cite{x509}

\paragraph{}
\textbf{Q5:}
\textit{Web servers offering weak cryptography are subject to several attacks. List 3 feasible attacks and explain them briefly. How did you configure your server to prevent such attacks?}

\begin{itemize}
    \item
        FREAK (Factoring Attack on RSA-EXPORT Keys)\cite{freak-ssl-tls}.
        A client using a vulnerable version of OpenSSL coupled with a server offering export-grade cryptography is exposed to a RSA-to-EXPORT\_RSA downgrade attack.
        The sesion key can then be brute-forced due to the shorter key length of EXPORT\_RSA.

\end{itemize}

All of these attacks are prevented by only allowing strong cipher suites in Apache.

\paragraph{}
\textbf{Q6:}
\cprotect\textit{Cookies can be a potential security risk if not handled properly, especially if they contain sensitive information. Two important flags can be set on cookies: \verb/HTTP-Only/ and \verb/secure/. What does the \verb/HTTP-Only/ flag do and how does it work? What does the \verb/secure/ flag do and how does it work?}

The \verb/HTTP-Only/ flag makes sure the cookie cannot be accessed from JavaScript. This stops malicious scripts from being able to obtain cookies from websites.

The \verb/secure/ flag ensures the cookie will only be transmitted to the server on HTTPS requests.

\paragraph{}
\textbf{Q7:}
\textit{What kind of malicious attacks is your web application (PHP) vulnerable to? Describe four briefly, and point out what countermeasures you have developed in your code to prevent such attacks.}

\textbf{SQL injections.} We make sure to use only prepared statements when interfacing the database.
\textbf{XSS}. All user input is escaped before being stored in the database.
\textbf{CSRF}. All forms contain a CSRF token.
\textbf{Cookies shared with other sites on the same domain}. No way to prevent this, because cookies cannot be restricted to only one port on the server.
The easiest way to fix this would be to use a custom domain for our site.
Another way would be using something other than cookies for storing session IDs, for instance JSON Web Tokens.

\paragraph{}
\textbf{Q8:}
\textit{Explain how you have stored the passwords of your users in your database. Why have you stored them like this? Why is it not a good idea to just store a users password as it is?}

The users' passwords have been hashed with the Blowfish\cite{blowfish} algorithm, and with a random salt per user.
The Blowfish algorithm is designed to consume a certain amount of time when hashing passwords, in contrast to for instance SHA1, which opts to use as little time as possible.
This makes it so that a brute force attack on the passwords will take a long time.

With a random salt for each user, techniques like rainbow tables, which are used to speed up a brute force attack, will not work as each password with a different salt would need its own rainbow table.

If the passwords were stored in plaintext, they would be readable by us and anyone who gains access to the database.
