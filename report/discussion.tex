\section{Discussion}

\subsection{Part I}

We created a certificate for our Apache Web server, signed by our very own Certificate Authority in the NTNU CA chain.
Being our own Certificate Authority allows us to issue new certificates when our empire needs expanding.

To achieve this, we registered our Group CA request at the course webpage, which was then signed by the course staff.
Our identities were confirmed to the course webpage by our personal .p12 files, signed by the NTNU Student CA.

\subsection{Part II}

The distributed versions of apache-httpd was verified, and installed.
Changes were made to the default SSL cipher suite to only allow high security SSL ciphers, as well as disabling the insecure hashing algorithm MD5.

\subsection{Part III}

After installing PHP, we made adjustments in the php.ini file to enhance the security of our application.
First, we configured PHP sessions to use the \verb/HTTP-Only/ and \verb/secure/ flags (explained in Q6).
We also changed the default session save path to a folder inside our home directory, to prevent other users of the same server to access our sessions.
Finally, we disabled public error logging (only logging to file), to minimize the amount of information others could get out of our application.

The biggest security issue we can think of in our implementation, is caused by the fact that our web site share a domain name with other web applications.
Unfortunately, the session cookie cannot be restricted to only a certain port, and this means other web sites on the same domain will also receive the session cookie when a user that is currently logged into our site sends requests to them.
This allows the websites to obtain session ids which can be used to authenticate as a user on our web site.
